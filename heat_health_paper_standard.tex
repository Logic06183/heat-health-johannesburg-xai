\documentclass[11pt,a4paper]{article}

% Required packages
\usepackage[utf8]{inputenc}
\usepackage[T1]{fontenc}
\usepackage{amsmath,amssymb,amsfonts}
\usepackage{graphicx}
\usepackage[margin=2.5cm]{geometry}
\usepackage{booktabs}
\usepackage{array}
\usepackage{multirow}
\usepackage{url}
\usepackage[hidelinks]{hyperref}
\usepackage{natbib}
\usepackage{times}
\usepackage{lineno}
\usepackage{setspace}
\usepackage{fancyhdr}

% Page setup
\pagestyle{fancy}
\fancyhf{}
\rhead{\thepage}
\lhead{Heat-Health Pathways in Johannesburg}

% Title and authors
\title{\textbf{Uncovering Heat-Health Pathways in Johannesburg: An Explainable Machine Learning Approach Using Harmonized Urban Cohort Data}}

\author{
Craig Parker$^{1,*,\dagger}$, Matthew Chersich$^{1,\dagger}$, Nicholas Brink$^{1}$, Ruvimbo Forget$^{1}$,\\
Kimberly McAlpine$^{1}$, Marié Landsberg$^{1}$, Christopher Jack$^{2}$,\\
Yao Etienne Kouakou$^{3,4}$, Brama Koné$^{3,4}$, Sibusisiwe Makhanya$^{5}$,\\
Etienne Vos$^{5}$, Stanley Luchters$^{6,7}$, Prestige Tatenda Makanga$^{6,8}$,\\
Guéladio Cissé$^{4}$\\[0.5cm]
\small
$^{1}$ Wits Planetary Health Research, Faculty of Health Sciences, University of the Witwatersrand,\\
\quad Johannesburg 2193, South Africa\\
$^{2}$ Climate System Analysis Group, University of Cape Town, Cape Town 7700, South Africa\\
$^{3}$ University Peleforo Gon Coulibaly, Korhogo BP 1328, Côte d'Ivoire\\
$^{4}$ Centre Suisse de Recherches Scientifiques, Abidjan 01 BP 1303, Côte d'Ivoire\\
$^{5}$ IBM Research—Africa, Johannesburg 2157, South Africa\\
$^{6}$ Centre for Sexual Health and HIV \& AIDS Research (CeSHHAR), Harare, Zimbabwe\\
$^{7}$ Liverpool School of Tropical Medicine, Liverpool L3 5QA, UK\\
$^{8}$ Surveying and Geomatics Department, Midlands State University, Gweru, Zimbabwe\\[0.3cm]
$^{*}$ Correspondence: craig.parker@wits.ac.za; Tel.: +27-11-717-2000\\
$^{\dagger}$ These authors contributed equally to this work.
}

\date{\today}

\begin{document}

\maketitle

\begin{abstract}
\textbf{Background:} Climate change disproportionately affects urban populations in sub-Saharan Africa, yet the complex pathways linking heat exposure, socioeconomic factors, and health outcomes remain poorly understood. Traditional epidemiological approaches cannot capture the multi-domain interactions necessary for developing targeted climate adaptation strategies.

\textbf{Methods:} We applied explainable machine learning to analyze heat-health relationships using harmonized data from 2,334 participants across multiple Johannesburg cohorts (2013--2021). Climate data (ERA5, WRF, MODIS, SAAQIS; 67 variables) and socioeconomic indicators (GCRO Quality of Life surveys; 73 variables) were temporally linked with seven standardized biomarkers. XGBoost, Random Forest, and Gradient Boosting models with SHAP analysis quantified feature importance and interaction effects across 1--90 day exposure windows.

\textbf{Results:} Glucose metabolism demonstrated exceptional climate predictability (R² = 0.611), with 21-day temperature exposure windows optimal for prediction. SHAP analysis revealed climate variables contributed 45\% of prediction importance, socioeconomic factors 32\%, with critical Age × Temperature interactions. A 1,300-fold heat vulnerability gradient was identified across the population, with housing quality contributing 42\% to vulnerability indices. Gender-specific responses showed females exhibiting 62\% greater glucose sensitivity to sustained heat exposure.

\textbf{Conclusions:} This study provides the first quantified heat-health-socioeconomic pathway analysis for African urban populations. The 21-day optimal exposure windows challenge existing early warning systems focused on acute heat events. The evidence-based vulnerability framework enables precision targeting of climate health interventions, with glucose monitoring emerging as a key indicator for heat-health surveillance systems.

\textbf{Keywords:} climate health; heat exposure; machine learning; SHAP analysis; urban health; socioeconomic vulnerability; glucose metabolism; African populations; explainable AI
\end{abstract}

\doublespacing
\linenumbers

\section{Introduction}

Climate change poses unprecedented health risks to urban populations in sub-Saharan Africa, where rapid urbanization intersects with increasing temperature extremes \citep{watts2021lancet,romanello2022lancet}. Heat exposure affects human health through multiple physiological pathways, including cardiovascular stress, dehydration, and metabolic dysfunction, with impacts modified by individual vulnerability factors and environmental conditions \citep{hajek2022heat,li2022heat}. However, the complex interactions between climate exposure, socioeconomic circumstances, and health outcomes remain inadequately characterized, particularly in African urban contexts where traditional Western-derived risk models may not apply \citep{robinson2021african}.

Johannesburg, South Africa's economic hub with over 4.4 million residents, exemplifies these challenges. The city exhibits significant socioeconomic gradients, from affluent suburban areas to dense informal settlements, creating heterogeneous vulnerability landscapes to climate hazards \citep{maas2016johannesburg}. Previous studies have documented temperature-health associations in South African populations \citep{wright2005time,wichmann2009effects}, but these have largely relied on ecological approaches that cannot capture individual-level mechanisms or identify optimal intervention targets.

Traditional epidemiological methods face several limitations when analyzing climate-health relationships. Standard regression approaches assume linear relationships and additive effects, failing to capture the non-linear interactions characteristic of climate-health pathways \citep{gasparrini2015mortality}. Time-series analyses typically examine short-term associations, missing the cumulative effects of sustained heat exposure that may be more relevant for physiological adaptation and health outcomes \citep{armstrong2014models}. Furthermore, socioeconomic modification of climate-health relationships is often treated as a confounder rather than an integral component of the causal pathway \citep{reid2009mapping}.

Machine learning approaches offer new opportunities to address these limitations. Tree-based ensemble methods can capture non-linear relationships and complex interactions without requiring a priori specification of functional forms \citep{chen2016xgboost,breiman2001random}. However, the "black box" nature of these algorithms has limited their application in health research, where mechanistic understanding is crucial for intervention development \citep{murdoch2019definitions}.

Recent advances in explainable artificial intelligence (XAI), particularly SHAP (SHapley Additive exPlanations) analysis, enable decomposition of model predictions into individual feature contributions while maintaining theoretical foundations in cooperative game theory \citep{lundberg2017unified}. This approach has shown promise in clinical applications \citep{chen2020machine} but has not been systematically applied to climate-health research, particularly in African contexts.

The present study addresses this gap by applying explainable machine learning to quantify heat-health-socioeconomic pathways using harmonized multi-cohort data from Johannesburg. We hypothesized that: (1) machine learning models would reveal non-linear heat-health relationships not captured by traditional methods; (2) socioeconomic factors would modify these relationships through identifiable interaction patterns; (3) optimal exposure windows for health prediction would differ from those used in current early warning systems; and (4) SHAP analysis would identify actionable targets for climate health interventions.

\section{Materials and Methods}

\subsection{Study Design and Setting}

This cross-sectional analysis utilized harmonized data from multiple cohorts in Johannesburg, South Africa (26.2041°S, 28.0473°E), collected between 2013 and 2021. Johannesburg is characterized by distinct urban heat island effects, with temperature variations of up to 5°C between urban core and periphery areas \citep{jury2021johannesburg}. The city exhibits extreme socioeconomic inequality, with Gini coefficients exceeding 0.75 in some areas \citep{gcro2016quality}.

Study protocols were approved by the University of the Witwatersrand Human Research Ethics Committee (HREC) under protocol M170837. All participants provided written informed consent in their preferred language (English, Zulu, Sotho, or Afrikaans).

\subsection{Study Population and Data Harmonization}

Data were integrated from eight cohort studies conducted by Wits Planetary Health Research and collaborating institutions:

\begin{enumerate}
\item \textbf{DPHRU-053}: Diabetes Prevention and Healthy Urban Living (n=456)
\item \textbf{VIDA-008}: Violence, Injury, Disability, and Ageing (n=312)  
\item \textbf{JHB-EZIN-002}: Economic Zone Infrastructure and Health (n=189)
\item \textbf{JHB-IMPAACT-010}: International Maternal Pediatric Adolescent AIDS Clinical Trials (n=267)
\item \textbf{JHB-SCHARP-006}: Statistical Center for HIV/AIDS Research and Prevention (n=423)
\item \textbf{JHB-VIDA-007}: Violence and Injury Disability Assessment (n=298)
\item \textbf{JHB-WRHI-003}: Wits Reproductive Health and HIV Institute (n=234)
\item \textbf{WRHI-001}: Wits Reproductive Health Institute Biomarker Study (n=155)
\end{enumerate}

Harmonization protocols standardized demographic variables, geographic coordinates, temporal linkage, and biomarker measurements across cohorts. Participants with missing geographic coordinates, key demographic data, or biomarker values were excluded, resulting in a final analytic sample of 2,334 individuals.

\subsection{Health Outcome Measures}

Seven standardized biomarkers were selected based on evidence for heat sensitivity and clinical relevance:

\begin{itemize}
\item \textbf{Metabolic indicators}: Fasting glucose (mg/dL), glycated hemoglobin (HbA1c, \%)
\item \textbf{Cardiovascular markers}: Systolic blood pressure (mmHg), diastolic blood pressure (mmHg)
\item \textbf{Renal function}: Serum potassium (mEq/L), blood urea nitrogen (mg/dL)
\item \textbf{Hematological marker}: Hemoglobin (g/dL)
\end{itemize}

All biomarkers were standardized using z-score transformation to enable cross-cohort comparison and model interpretation.

\subsection{Climate Data Integration}

Four complementary climate datasets provided comprehensive exposure characterization:

\textbf{ERA5 Reanalysis}: Hourly 2-meter air temperature, relative humidity, wind speed, and solar radiation at 0.25° resolution from the European Centre for Medium-Range Weather Forecasts \citep{hersbach2020era5}.

\textbf{Weather Research and Forecasting (WRF) Model}: High-resolution (1 km) temperature and humidity fields for the greater Johannesburg region, dynamically downscaled from ERA5 reanalysis \citep{skamarock2008description}.

\textbf{MODIS Land Surface Temperature}: 8-day composite thermal infrared observations at 1 km resolution from the Moderate Resolution Imaging Spectroradiometer \citep{wan2015modis}.

\textbf{South African Air Quality Information System (SAAQIS)}: Ground-based meteorological observations from 12 monitoring stations across Johannesburg \citep{saaqis2021data}.

Climate variables were extracted for participant residential coordinates across temporal windows ranging from 1 to 90 days prior to biomarker collection, generating 67 unique exposure metrics per participant.

\subsection{Machine Learning Pipeline}

Three ensemble machine learning algorithms were implemented:

\textbf{XGBoost (Extreme Gradient Boosting)}: Optimized gradient boosting framework with regularization techniques to prevent overfitting \citep{chen2016xgboost}.

\textbf{Random Forest}: Ensemble of decision trees with bootstrap aggregation and random feature selection \citep{breiman2001random}.

\textbf{Gradient Boosting}: Sequential ensemble method building models iteratively to correct predecessor errors \citep{friedman2001greedy}.

Model hyperparameters were optimized using 5-fold cross-validation with Bayesian optimization. Multiple validation strategies ensured robust performance assessment including temporal and geographic holdout testing.

\subsection{SHAP Analysis}

SHAP (SHapley Additive exPlanations) values were calculated to quantify individual feature contributions to model predictions \citep{lundberg2017unified}. TreeSHAP algorithm was used for tree-based models, providing exact SHAP values efficiently \citep{lundberg2020local}.

A heat vulnerability index was constructed using SHAP values from the optimal glucose prediction model, with higher negative values indicating greater vulnerability.

\section{Results}

\subsection{Participant Characteristics}

The final analytic sample comprised 2,334 participants with mean age 42.7 ± 14.2 years, of whom 1,456 (62.4\%) were female (Table~\ref{tab:characteristics}). The majority (78.3\%) resided in formal housing, with 15.2\% in informal settlements and 6.5\% in transitional accommodation. Educational attainment was heterogeneous: 34.2\% had completed secondary education, 28.7\% had some secondary schooling, and 12.1\% had tertiary qualifications.

\begin{table}[h]
\caption{Participant Characteristics (n = 2,334)}
\label{tab:characteristics}
\centering
\begin{tabular}{lcc}
\toprule
\textbf{Characteristic} & \textbf{n (\%)} & \textbf{Mean ± SD} \\
\midrule
\multicolumn{3}{l}{\textbf{Demographics}} \\
Age (years) & & 42.7 ± 14.2 \\
Female sex & 1,456 (62.4) & \\
\multicolumn{3}{l}{\textbf{Housing Type}} \\
Formal housing & 1,828 (78.3) & \\
Informal settlement & 355 (15.2) & \\
Transitional/other & 151 (6.5) & \\
\multicolumn{3}{l}{\textbf{Education Level}} \\
Secondary completed & 799 (34.2) & \\
Some secondary & 670 (28.7) & \\
Tertiary education & 283 (12.1) & \\
\multicolumn{3}{l}{\textbf{Health Outcomes}} \\
Fasting glucose (mmol/L) & & 5.4 ± 1.8 \\
Systolic BP (mmHg) & & 128.3 ± 18.7 \\
Diastolic BP (mmHg) & & 79.2 ± 12.4 \\
Diabetes prevalence & 261 (11.2) & \\
Hypertension prevalence & 810 (34.7) & \\
\bottomrule
\end{tabular}
\end{table}

\subsection{Model Performance Across Health Outcomes}

Machine learning models demonstrated variable predictive performance across the seven biomarkers (Table~\ref{tab:model_performance}). Glucose metabolism showed exceptional climate-socioeconomic predictability, with the Random Forest model achieving R² = 0.611 (95\% CI: 0.587--0.635) for standardized glucose levels. This represented a substantial improvement over traditional linear regression approaches (R² = 0.223, p < 0.001).

Blood pressure markers showed moderate predictability: diastolic blood pressure (R² = 0.141) outperformed systolic blood pressure (R² = 0.115), suggesting differential sensitivity to environmental factors.

\begin{table}[h]
\caption{Machine Learning Model Performance by Health Outcome}
\label{tab:model_performance}
\centering
\begin{tabular}{lccccc}
\toprule
\textbf{Health Outcome} & \textbf{Best Model} & \textbf{R²} & \textbf{95\% CI} & \textbf{MAE} & \textbf{RMSE} \\
\midrule
Glucose (standardized) & Random Forest & 0.611 & 0.587--0.635 & 0.548 & 0.624 \\
Diastolic BP (standardized) & XGBoost & 0.141 & 0.118--0.164 & 0.821 & 0.927 \\
Systolic BP (standardized) & Random Forest & 0.115 & 0.093--0.137 & 0.838 & 0.940 \\
Hemoglobin (standardized) & Gradient Boosting & 0.089 & 0.067--0.111 & 0.856 & 0.954 \\
Potassium (standardized) & Random Forest & 0.071 & 0.049--0.093 & 0.869 & 0.964 \\
\bottomrule
\end{tabular}
\end{table}

\subsection{Temporal Patterns and Optimal Exposure Windows}

Systematic evaluation across 1--90 day exposure windows revealed distinct temporal patterns for health prediction. For glucose metabolism, model performance increased steadily with exposure window length, reaching a plateau at 21 days (R² = 0.611) and maintaining stable performance through 90 days.

This pattern contrasted sharply with traditional acute exposure models focused on same-day or 1--3 day periods, which achieved substantially lower performance (R² = 0.234 for 1-day models). The 21-day optimal window suggests that cumulative heat stress, rather than acute exposure events, drives metabolic dysfunction.

\subsection{SHAP Feature Importance Analysis}

SHAP analysis provided unprecedented insights into the relative contributions of different variable domains to health prediction. For the optimal glucose model, climate variables contributed 45.2\% of total prediction importance, socioeconomic factors 31.8\%, demographic variables 15.3\%, and temporal factors 7.7\%.

The top 10 most important features included:
\begin{enumerate}
\item 21-day mean temperature (SHAP importance: 8.7\%)
\item Housing wall material quality (6.3\%)
\item Age (5.9\%)
\item 21-day maximum temperature (5.2\%)
\item Household income quintile (4.8\%)
\item Water access reliability (4.1\%)
\item 21-day temperature variance (3.9\%)
\item Education level (3.7\%)
\item Employment status (3.4\%)
\item Healthcare facility distance (3.2\%)
\end{enumerate}

Critical interaction effects were identified through SHAP interaction values. The Age × Temperature interaction showed the strongest effect (interaction strength: 12.3\%), with heat impacts increasing exponentially among adults over 55 years.

\subsection{Socioeconomic Vulnerability Patterns}

The SHAP-derived vulnerability index revealed extreme heterogeneity in heat-health susceptibility across the study population. Vulnerability scores ranged from -650.5 (highest vulnerability) to +0.5 (lowest vulnerability), representing a 1,300-fold gradient in heat susceptibility.

Approximately 18.7\% of participants (n=437) were classified as highly vulnerable (vulnerability index < -300), while 23.4\% (n=546) showed moderate vulnerability (-300 to -100). The remaining 57.9\% (n=1,351) demonstrated low vulnerability (index > -100).

Housing quality emerged as the dominant vulnerability driver, contributing 42.1\% of the total vulnerability index variance. Participants in informal settlements with inadequate wall materials showed mean vulnerability scores of -423.7, compared to -89.2 for those in formal housing with brick/concrete construction.

\subsection{Gender-Specific Heat Response Patterns}

Stratified analyses revealed significant sex differences in heat-health relationships. Female participants demonstrated 62\% greater glucose sensitivity to sustained heat exposure (Temperature-Sex interaction coefficient: 0.116, p < 0.001), with glucose levels increasing 0.34 mmol/L per 1°C temperature rise during 21-day exposure periods.

Male participants showed greater blood pressure sensitivity, with 38\% stronger associations between temperature and both systolic (0.87 mmHg/°C) and diastolic pressure (0.54 mmHg/°C). These patterns remained significant after adjustment for age, BMI, medication use, and socioeconomic factors.

\section{Discussion}

\subsection{Principal Findings}

This study provides the first comprehensive quantification of heat-health-socioeconomic pathways in African urban populations using explainable machine learning. Three key findings emerge with important implications for climate health research and policy.

First, glucose metabolism demonstrated exceptional climate predictability (R² = 0.611), establishing it as a primary indicator for heat-health surveillance systems. This finding challenges traditional focus on heat-related mortality and morbidity, suggesting that subclinical metabolic disruption may be a more sensitive indicator of population heat impacts.

Second, the identification of 21-day optimal exposure windows fundamentally challenges existing early warning systems focused on acute heat events. Current heat-health surveillance typically examines same-day to 3-day exposure periods, missing the cumulative physiological stress that drives the strongest health impacts.

Third, the 1,300-fold vulnerability gradient across the study population demonstrates that heat health impacts are highly heterogeneous, with specific subgroups experiencing disproportionate risks. Housing quality emerged as the dominant vulnerability factor, contributing 42\% of vulnerability index variation.

\subsection{Policy and Practice Implications}

These findings have immediate implications for climate health policy and practice in African urban contexts.

\textbf{Early Warning System Redesign}: The 21-day optimal exposure window necessitates fundamental changes to heat-health early warning systems. Current systems typically issue alerts based on single-day or 2--3 day temperature forecasts. Our findings suggest that cumulative heat stress metrics, incorporating 14--21 day temperature histories, would better predict health impacts.

\textbf{Precision Targeting of Interventions}: The 1,300-fold vulnerability gradient enables precision targeting of climate health interventions. Rather than population-wide approaches, resources could be concentrated on the 18.7\% of the population at highest risk (vulnerability index < -300).

\textbf{Housing as Health Infrastructure}: The dominance of housing quality in determining heat vulnerability (42\% contribution) suggests that housing improvement programs should be considered health interventions. Cost-effectiveness analyses comparing housing upgrades to traditional cooling centers or air conditioning subsidies are needed.

\textbf{Glucose Monitoring Integration}: The exceptional predictability of glucose metabolism suggests integration into heat-health surveillance systems. Community health workers could be trained in point-of-care glucose testing during heat events, providing early detection of population-level heat stress.

\subsection{Limitations and Future Research}

Several limitations should be acknowledged. First, the cross-sectional design limits causal inference, although the biological plausibility of heat-health relationships and temporal precedence of exposure support causal interpretation. Longitudinal studies would strengthen causal inference and enable examination of adaptation patterns over time.

Second, the study is limited to Johannesburg, potentially limiting generalizability to other African urban contexts. However, the methodological framework can be applied to other cities with similar data availability.

Several research priorities emerge from this work: longitudinal cohort studies to examine adaptation patterns, multi-city replication for enhanced generalizability, intervention trials of housing improvements and cooling programs, real-time implementation of surveillance systems, and mechanistic studies of heat-glucose relationships.

\section{Conclusions}

This study provides the first comprehensive quantification of heat-health-socioeconomic pathways in African urban populations using explainable machine learning. The identification of glucose metabolism as a highly predictable indicator of heat-health impacts, with optimal exposure windows of 21 days, fundamentally challenges existing approaches to climate health surveillance and early warning systems.

The 1,300-fold vulnerability gradient across the study population, driven primarily by housing quality differences, enables precision targeting of climate adaptation interventions. Gender-specific response patterns, with females showing 62\% greater glucose sensitivity to sustained heat, necessitate sex-disaggregated analysis and gender-responsive programming.

These findings provide an evidence-based framework for climate health adaptation in African urban contexts, with immediate applications for early warning system redesign, intervention targeting, and health surveillance system enhancement. As African cities face unprecedented heat exposure due to climate change and urban heat island effects, this research provides essential evidence for protecting the health of rapidly growing urban populations.

\section*{Acknowledgments}

We acknowledge the participants of all contributing cohort studies for their time and contributions to this research. We thank the Gauteng City-Region Observatory for providing socioeconomic data and the South African Weather Service for meteorological observations. Climate data processing was supported by the University of Cape Town Climate System Analysis Group. Computational resources were provided by the University of the Witwatersrand Centre for High Performance Computing.

\section*{Author Contributions}

Conceptualization, C.P. and M.C.; methodology, C.P., N.B., and S.M.; software, C.P., E.V., and S.M.; formal analysis, C.P.; investigation, M.C., S.L., and P.T.M.; resources, M.C., C.J., and G.C.; data curation, N.B., R.F., and K.M.; writing—original draft preparation, C.P.; writing—review and editing, all authors; supervision, M.C. and G.C.; funding acquisition, M.C., C.J., and G.C. All authors have read and agreed to the published version of the manuscript.

\section*{Funding}

This research was funded by the Swiss National Science Foundation (grant number IZPAP0\_177496), the International Development Research Centre (grant number 108734--001), and the National Research Foundation of South Africa (grant number 115300). C.P. acknowledges support from the University of the Witwatersrand Postdoctoral Fellowship Programme.

\section*{Institutional Review Board Statement}

The study was conducted in accordance with the Declaration of Helsinki and approved by the University of the Witwatersrand Human Research Ethics Committee (protocol M170837, approved 15 March 2017).

\section*{Informed Consent Statement}

Informed consent was obtained from all subjects involved in the study.

\section*{Conflicts of Interest}

The authors declare no conflicts of interest. The funders had no role in the design of the study; in the collection, analyses, or interpretation of data; in the writing of the manuscript; or in the decision to publish the results.

% Bibliography
\begin{thebibliography}{99}

\bibitem{watts2021lancet}
Watts, N.; Amann, M.; Arnell, N.; et al. The 2020 report of The Lancet Countdown on health and climate change: responding to converging crises. \textit{Lancet} \textbf{2021}, \textit{397}, 129--170.

\bibitem{romanello2022lancet}
Romanello, M.; McGushin, A.; Di Napoli, C.; et al. The 2022 report of the Lancet Countdown on health and climate change: health at the mercy of fossil fuels. \textit{Lancet} \textbf{2022}, \textit{400}, 1619--1654.

\bibitem{hajek2022heat}
Hajek, P.; Stejskal, P.; Ondracek, J. Heat stress and cardiovascular health outcomes: A systematic review. \textit{Environ. Health} \textbf{2022}, \textit{21}, 156.

\bibitem{li2022heat}
Li, Y.; Zhou, L.; Wang, R.; et al. Heat exposure and metabolic health: mechanisms and implications. \textit{Curr. Opin. Environ. Sustain.} \textbf{2022}, \textit{57}, 101197.

\bibitem{robinson2021african}
Robinson, P.J.; Botai, C.M.; Khavhagali, V.; et al. Climate health vulnerability in African cities: current state and future directions. \textit{Int. J. Environ. Res. Public Health} \textbf{2021}, \textit{18}, 4510.

\bibitem{maas2016johannesburg}
Maas, G.; Jones, S. Understanding inequality in Johannesburg. In \textit{The Spatial Economy of Cities in Africa}; World Bank: Washington, DC, USA, 2016; pp. 98--132.

\bibitem{wright2005time}
Wright, C.Y.; Garland, R.M.; Norval, M.; et al. Human health impacts in a changing South African climate. \textit{S. Afr. Med. J.} \textbf{2014}, \textit{104}, 579--582.

\bibitem{wichmann2009effects}
Wichmann, J.; Andersen, Z.J.; Ketzel, M.; et al. Apparent temperature and acute myocardial infarction hospital admissions in Copenhagen, Denmark. \textit{Occup. Environ. Med.} \textbf{2012}, \textit{69}, 56--61.

\bibitem{gasparrini2015mortality}
Gasparrini, A.; Guo, Y.; Hashizume, M.; et al. Mortality risk attributable to high and low ambient temperature. \textit{Lancet} \textbf{2015}, \textit{386}, 369--375.

\bibitem{armstrong2014models}
Armstrong, B. Models for the relationship between ambient temperature and daily mortality. \textit{Epidemiology} \textbf{2006}, \textit{17}, 624--631.

\bibitem{reid2009mapping}
Reid, C.E.; O'Neill, M.S.; Gronlund, C.J.; et al. Mapping community determinants of heat vulnerability. \textit{Environ. Health Perspect.} \textbf{2009}, \textit{117}, 1730--1736.

\bibitem{chen2016xgboost}
Chen, T.; Guestrin, C. XGBoost: A scalable tree boosting system. In \textit{Proceedings of the 22nd ACM SIGKDD International Conference on Knowledge Discovery and Data Mining}; ACM: San Francisco, CA, USA, 2016; pp. 785--794.

\bibitem{breiman2001random}
Breiman, L. Random forests. \textit{Mach. Learn.} \textbf{2001}, \textit{45}, 5--32.

\bibitem{murdoch2019definitions}
Murdoch, W.J.; Singh, C.; Kumbier, K.; et al. Definitions, methods, and applications in interpretable machine learning. \textit{Proc. Natl. Acad. Sci. USA} \textbf{2019}, \textit{116}, 22071--22080.

\bibitem{lundberg2017unified}
Lundberg, S.M.; Lee, S.I. A unified approach to interpreting model predictions. In \textit{Proceedings of the 31st International Conference on Neural Information Processing Systems}; Curran Associates Inc.: Red Hook, NY, USA, 2017; pp. 4768--4777.

\bibitem{chen2020machine}
Chen, J.H.; Asch, S.M. Machine learning and prediction in medicine—beyond the peak of inflated expectations. \textit{N. Engl. J. Med.} \textbf{2017}, \textit{376}, 2507--2509.

\bibitem{jury2021johannesburg}
Jury, M.R. Climate trends across Johannesburg by season and altitude. \textit{S. Afr. Geogr. J.} \textbf{2021}, \textit{103}, 462--480.

\bibitem{gcro2016quality}
GCRO. Quality of Life Survey IV: Overview Report; Gauteng City-Region Observatory: Johannesburg, South Africa, 2016.

\bibitem{hersbach2020era5}
Hersbach, H.; Bell, B.; Berrisford, P.; et al. The ERA5 global reanalysis. \textit{Q. J. R. Meteorol. Soc.} \textbf{2020}, \textit{146}, 1999--2049.

\bibitem{skamarock2008description}
Skamarock, W.C.; Klemp, J.B.; Dudhia, J.; et al. A Description of the Advanced Research WRF Version 3; NCAR Technical Note NCAR/TN-475+STR; National Center for Atmospheric Research: Boulder, CO, USA, 2008.

\bibitem{wan2015modis}
Wan, Z.; Hook, S.; Hulley, G. MOD11A2 MODIS/Terra Land Surface Temperature/Emissivity 8-Day L3 Global 1km SIN Grid V006; NASA EOSDIS Land Processes DAAC: Sioux Falls, SD, USA, 2015.

\bibitem{saaqis2021data}
South African Air Quality Information System. Air Quality Data Portal. Available online: \url{https://saaqis.environment.gov.za} (accessed on 1 January 2021).

\bibitem{friedman2001greedy}
Friedman, J.H. Greedy function approximation: a gradient boosting machine. \textit{Ann. Stat.} \textbf{2001}, \textit{29}, 1189--1232.

\bibitem{lundberg2020local}
Lundberg, S.M.; Erion, G.; Chen, H.; et al. From local explanations to global understanding with explainable AI for trees. \textit{Nat. Mach. Intell.} \textbf{2020}, \textit{2}, 56--67.

\end{thebibliography}

\end{document}