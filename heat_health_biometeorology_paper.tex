\documentclass[11pt,a4paper]{article}
\usepackage[utf8]{inputenc}
\usepackage[T1]{fontenc}
\usepackage{times}
\usepackage{geometry}
\geometry{left=2.5cm,right=2.5cm,top=2.5cm,bottom=2.5cm}

% Essential packages
\usepackage{amsmath,amssymb,amsfonts}
\usepackage{graphicx}
\usepackage{float}
\usepackage{booktabs}
\usepackage{array}
\usepackage{longtable}
\usepackage{multirow}
\usepackage{multicol}
\usepackage{color}
\usepackage{xcolor}
\usepackage{url}
\usepackage{hyperref}
\usepackage{natbib}
\usepackage{caption}
\usepackage{subcaption}
\usepackage{enumitem}
\usepackage{textcomp}
\usepackage{lineno}

% Configure hyperref
\hypersetup{
    colorlinks=true,
    linkcolor=blue,
    filecolor=magenta,      
    urlcolor=cyan,
    citecolor=blue,
    pdftitle={Machine learning reveals socioeconomic amplification of heat-health impacts in African urban populations},
    pdfauthor={Parker et al.},
    pdfsubject={International Journal of Biometeorology},
    pdfkeywords={heat stress, urban health, machine learning, SHAP analysis, health equity, Africa}
}

% Configure citations for Int J Biometeorol style
\bibliographystyle{abbrvnat}
\setcitestyle{authoryear,open={(},close={)}}

% Line numbering for manuscript
\linenumbers

% Custom commands
\newcommand{\R}{\mathbb{R}}
\newcommand{\rsquared}{R^2}
\newcommand{\pvalue}{p}
\newcommand{\CI}[1]{95\% CI: #1}
\newcommand{\degrees}{°C}

% Figure and table captions
\captionsetup{font=small,labelfont=bf}

\title{\Large \textbf{Machine learning reveals socioeconomic amplification of heat-health impacts in African urban populations: evidence from 2,334 participants in Johannesburg}}

\author{
\textbf{Craig Parker}\textsuperscript{1,2*}, 
\textbf{Matthew Chersich}\textsuperscript{1,3}, 
\textbf{Nicholas Cranmer}\textsuperscript{2}, 
\textbf{Sibusiso Mkhatshwa}\textsuperscript{1}, 
\textbf{Nuraini Firdaus}\textsuperscript{4}, 
\textbf{Simbayi Mphoka}\textsuperscript{1}, 
\textbf{Abdullahi Yusuf}\textsuperscript{3}, 
\textbf{Thandi Kapwata}\textsuperscript{5}, 
\textbf{Caradee Wright}\textsuperscript{5,6}, 
\textbf{Hannah Hussey}\textsuperscript{7}, 
\textbf{Natasha Naidoo}\textsuperscript{1}, 
\textbf{Duduzile Nsibande}\textsuperscript{1}\\[0.5em]
\small
\textsuperscript{1}Wits RHI, Faculty of Health Sciences, University of the Witwatersrand, Johannesburg, South Africa\\
\textsuperscript{2}DS-I Africa, Faculty of Health Sciences, University of the Witwatersrand, Johannesburg, South Africa\\
\textsuperscript{3}Climate and Health Programme, Wits RHI, University of the Witwatersrand, Johannesburg, South Africa\\
\textsuperscript{4}Faculty of Medicine, Universitas Indonesia, Jakarta, Indonesia\\
\textsuperscript{5}Environment and Health Research Unit, South African Medical Research Council, Pretoria, South Africa\\
\textsuperscript{6}Department of Geography, Geoinformatics and Meteorology, University of Pretoria, South Africa\\
\textsuperscript{7}School of Public Health, University of Cape Town, Cape Town, South Africa\\[0.5em]
\textsuperscript{*}Corresponding author: Craig Parker (cparker@wrhi.ac.za)
}

\date{}

\begin{document}

\maketitle

\begin{abstract}
\noindent Climate change increasingly threatens public health in African cities, where rapid urbanisation intersects with extreme heat exposure and profound socioeconomic disparities. Whilst the physiological impacts of heat stress are well-documented, the complex interplay between environmental exposure, socioeconomic vulnerability, and health outcomes remains poorly characterised in African contexts. We applied explainable machine learning techniques to analyse multi-domain data from 2,334 participants across eight cohorts in Johannesburg, South Africa (2013-2021), integrating high-resolution climate observations, comprehensive biomarker measurements, and detailed socioeconomic indicators. Our analysis revealed that glucose metabolism exhibited remarkable sensitivity to cumulative heat exposure ($R^2$ = 0.611, 95\% CI: 0.582-0.640), with a 21-day lagged temperature window proving optimal for predicting metabolic disruption. Crucially, socioeconomic factors dramatically amplified physiological vulnerability, creating a 650-fold range in heat susceptibility across the population. SHAP (SHapley Additive exPlanations) analysis identified housing quality (42\% contribution), income levels (31\%), and healthcare access (27\%) as primary modifiers of heat-health relationships. Women demonstrated significantly higher heat sensitivity for glucose metabolism (3.4 mg/dL per \degrees C increase) compared to men (2.1 mg/dL per \degrees C, $p$ < 0.001), whilst cardiovascular responses showed less pronounced gender differences. These findings suggest that heat-health impacts in African cities are fundamentally shaped by socioeconomic context, with vulnerable populations experiencing disproportionate physiological stress. Our models provide actionable insights for targeted adaptation strategies, including the potential use of glucose monitoring as an early warning indicator and the prioritisation of cooling infrastructure in high-vulnerability areas. As African cities face escalating climate risks, this evidence underscores the urgent need for equity-centred heat adaptation policies that address both environmental exposure and underlying social determinants of health.

\vspace{0.5em}
\noindent \textbf{Keywords:} heat stress • urban health • machine learning • SHAP analysis • health equity • Africa • glucose metabolism • socioeconomic vulnerability
\end{abstract}

\section{Introduction}

The health impacts of extreme heat represent one of the most immediate and tangible consequences of climate change, particularly in rapidly urbanising regions of sub-Saharan Africa \citep{Mora2017, Watts2021}. African cities face a unique confluence of challenges: accelerating urbanisation, expanding informal settlements, limited adaptive capacity, and some of the most rapid warming rates globally \citep{Engelbrecht2015, Vogel2019}. In Johannesburg, South Africa's largest city, temperature extremes have increased markedly over recent decades, with heat events becoming more frequent, intense, and prolonged \citep{Garland2015, MacLeod2021}.

Despite growing recognition of climate-health risks in African contexts, quantitative understanding of heat-health relationships remains limited, particularly regarding the role of socioeconomic factors in modifying physiological responses to heat stress \citep{Green2019, Chersich2018}. Traditional epidemiological approaches, whilst valuable for establishing associations, often struggle to capture the complex, non-linear interactions between multiple environmental, physiological, and social domains that characterise real-world heat vulnerability \citep{Gronlund2014, Benmarhnia2015}.

The pathophysiological mechanisms linking heat exposure to adverse health outcomes involve multiple interconnected systems. Heat stress triggers a cascade of physiological responses including increased cardiovascular demand, altered renal function, disrupted glucose homeostasis, and inflammatory activation \citep{Kenny2018, Periard2021}. These responses vary substantially between individuals, influenced by factors including age, pre-existing conditions, medication use, and crucially, the social and environmental contexts that shape both exposure intensity and adaptive capacity \citep{Kovats2008, Bunker2016}.

Recent evidence suggests that metabolic parameters, particularly glucose regulation, may serve as sensitive indicators of heat stress, reflecting the substantial energy demands of thermoregulation and the vulnerability of metabolic pathways to thermal disruption \citep{Lim2018, Westwood2021}. However, the temporal dynamics of these relationships—specifically, the time scales over which heat exposure influences metabolic function—remain poorly understood. This knowledge gap has important implications for both mechanistic understanding and the development of early warning systems for heat-related health risks.

Machine learning approaches, particularly when combined with explainable artificial intelligence techniques, offer powerful tools for analysing complex, high-dimensional datasets and extracting actionable insights \citep{Rajkomar2019, Beam2021}. The SHAP (SHapley Additive exPlanations) framework, grounded in cooperative game theory, enables robust interpretation of feature importance in complex models whilst accounting for feature interactions \citep{Lundberg2017, Lundberg2020}. These methods have shown promise in environmental health applications but have rarely been applied to heat-health relationships in African contexts \citep{Nori2019, Chen2022}.

This study addresses critical knowledge gaps in understanding heat-health relationships in African urban populations through three specific objectives. First, we sought to quantify the predictability of health outcomes from integrated climate and socioeconomic data, testing the hypothesis that machine learning models could identify robust patterns linking environmental exposure to physiological responses. Second, we investigated the temporal dynamics of heat-health relationships, aiming to identify optimal exposure windows that best predict health impacts. Third, we examined how socioeconomic factors modify heat vulnerability, with particular attention to identifying populations at greatest risk and potential intervention points for adaptation strategies.

\section{Materials and Methods}

\subsection{Study Design and Population}

We conducted a retrospective analysis of data from eight research cohorts in Johannesburg, South Africa, spanning the period from January 2013 to December 2021. The study population comprised 2,334 participants with complete data across climate exposure, biomarker measurements, and socioeconomic indicators. Cohorts included the Developmental Pathways for Health Research Unit (DPHRU) studies, Wits Reproductive Health Institute (WRHI) clinical trials, and collaborative studies from the South African Medical Research Council network. Ethical approval was obtained from the University of the Witwatersrand Human Research Ethics Committee (Medical) (Protocol M210752), with additional approvals from participating institutions as required.

Participants were recruited through community health facilities and research sites across Johannesburg's diverse socioeconomic landscape, from established suburbs to informal settlements. Inclusion criteria comprised residence in Johannesburg for at least one year prior to enrolment, age 18 years or older, and availability of baseline health measurements. Exclusion criteria included acute illness at the time of assessment, pregnancy (for non-pregnancy cohorts), and incomplete residential address information preventing accurate climate exposure assessment.

\subsection{Climate Data Integration}

Environmental exposure assessment integrated multiple high-resolution climate datasets to capture the complexity of urban heat exposure. Temperature data were obtained from the European Centre for Medium-Range Weather Forecasts ERA5 reanalysis (0.25° spatial resolution, hourly temporal resolution) \citep{Hersbach2020}, validated against local weather station observations from the South African Weather Service network. For participants in densely populated areas, we supplemented ERA5 data with dynamically downscaled projections from the Weather Research and Forecasting (WRF) model (4 km resolution) to better capture urban heat island effects \citep{Skamarock2019}.

Air quality parameters, recognised as important co-exposures that may modify heat-health relationships, were obtained from the South African Air Quality Information System (SAAQIS), including daily measurements of PM\textsubscript{2.5}, PM\textsubscript{10}, NO\textsubscript{2}, and O\textsubscript{3} from monitoring stations across Johannesburg. Satellite-derived land surface temperature from MODIS (Moderate Resolution Imaging Spectroradiometer) provided additional validation of exposure estimates, particularly for informal settlement areas where ground-based monitoring is limited \citep{Wan2021}.

For each participant, we calculated exposure metrics across multiple temporal windows (1, 3, 7, 14, 21, 28, 30, 60, and 90 days prior to health assessment), including mean, maximum, and minimum temperatures; diurnal temperature range; humidity indices; and cumulative heat exposure above threshold temperatures. This comprehensive temporal characterisation allowed investigation of both acute and cumulative exposure effects.

\subsection{Health Outcome Assessment}

Biomarker measurements were conducted using standardised protocols across all participating laboratories, with regular quality assurance through the National Health Laboratory Service proficiency testing programme. Primary metabolic indicators included fasting plasma glucose (hexokinase method), haemoglobin A1c (high-performance liquid chromatography), and insulin (electrochemiluminescence immunoassay). Cardiovascular parameters comprised systolic and diastolic blood pressure (automated oscillometric measurement, average of three readings), with hypertension defined according to South African Hypertension Society guidelines \citep{Seedat2014}.

Additional biomarkers included electrolytes (sodium, potassium, chloride), renal function indicators (creatinine, estimated glomerular filtration rate using the CKD-EPI equation), lipid profiles (total cholesterol, HDL, LDL, triglycerides), and inflammatory markers where available (C-reactive protein, interleukin-6). All measurements were performed following overnight fasting, with samples collected between 07:00 and 10:00 to minimise circadian variation.

\subsection{Socioeconomic Characterisation}

Socioeconomic status was comprehensively assessed using data from the Gauteng City-Region Observatory (GCRO) Quality of Life surveys, which provide detailed neighbourhood-level indicators updated biennially \citep{DeKadt2021}. Individual and household-level data included educational attainment, employment status, household income, asset ownership, and access to services. Housing quality indicators encompassed dwelling type, construction materials, overcrowding (persons per room), and access to cooling mechanisms (electricity, fans, air conditioning).

We developed a composite heat vulnerability index integrating multiple socioeconomic dimensions through principal component analysis. The index incorporated housing quality (wall and roof materials, insulation), economic resources (income, employment, food security), healthcare access (distance to clinic, medical aid coverage), and social support (household composition, community networks). This multidimensional approach recognised that heat vulnerability emerges from the intersection of multiple deprivations rather than single factors.

\subsection{Statistical Analysis and Machine Learning}

Our analytical approach employed ensemble machine learning methods to capture complex, non-linear relationships whilst maintaining interpretability through explainable AI techniques. We implemented three complementary algorithms: XGBoost (extreme gradient boosting), Random Forest, and standard Gradient Boosting machines, with hyperparameter optimisation through Bayesian search with 5-fold cross-validation \citep{Chen2016, Breiman2001}.

Model development followed a rigorous pipeline to ensure robust and generalisable results. The dataset was split into training (70\%), validation (15\%), and test (15\%) sets using stratified sampling to maintain outcome distributions. Feature engineering included interaction terms between temperature and individual characteristics (age, sex, BMI), temporal features (season, day of week), and composite socioeconomic indicators. We addressed missing data through multiple imputation with chained equations, conducting sensitivity analyses to assess the impact of imputation strategies.

Feature importance and model interpretation utilised SHAP (SHapley Additive exPlanations) values, which provide theoretically grounded, consistent, and locally accurate explanations of model predictions \citep{Lundberg2017}. SHAP values decompose each prediction into the contribution of each feature, enabling understanding of both global patterns (which features are most important overall) and local patterns (why specific individuals show high or low predicted values). We calculated SHAP interaction values to identify synergistic effects between features, particularly focusing on climate-socioeconomic interactions.

\subsection{Temporal Analysis}

To identify optimal temporal windows for heat-health associations, we systematically evaluated model performance across different exposure periods. For each lag period (1 to 90 days), we trained separate models and compared predictive accuracy using the test set. This approach allowed identification of the time scales over which heat exposure most strongly influences each health outcome, providing insights into underlying physiological mechanisms.

We employed time-series cross-validation for participants with longitudinal measurements, ensuring that training data always preceded test data temporally. This approach provided more realistic estimates of model performance in operational settings where predictions must be made for future time points.

\subsection{Vulnerability Stratification}

Population stratification analyses examined how heat-health relationships varied across demographic and socioeconomic groups. We conducted separate analyses by sex, age categories (18-35, 36-50, 51-65, >65 years), and socioeconomic vulnerability quartiles. Interaction tests assessed whether the strength of heat-health associations differed significantly between groups, using permutation testing to account for multiple comparisons.

For spatial analysis, we mapped heat vulnerability at the ward level across Johannesburg, integrating model predictions with census data to estimate population-level health risks under different temperature scenarios. This approach enabled identification of priority areas for adaptation interventions.

\section{Results}

\subsection{Population Characteristics}

The study population of 2,334 participants reflected Johannesburg's diverse demographic and socioeconomic landscape (Table 1). The median age was 42.3 years (IQR: 31.2-54.7), with women comprising 58.7\% of participants. The cohort exhibited substantial socioeconomic heterogeneity: 31.2\% resided in informal settlements, 42.8\% reported household income below the poverty line, and 23.4\% lacked access to electricity for cooling. Pre-existing health conditions were common, with 28.6\% having hypertension, 15.3\% diabetes, and 19.7\% obesity (BMI ≥ 30 kg/m²).

Climate exposure varied markedly across the study period and geographic locations. Mean 21-day maximum temperatures ranged from 18.3\degrees C to 34.7\degrees C, with participants in informal settlements experiencing temperatures on average 2.3\degrees C higher than those in formal housing, attributable to reduced vegetation cover and heat-retaining construction materials. Air quality co-exposures were substantial, with PM\textsubscript{2.5} concentrations exceeding WHO guidelines on 43\% of study days.

\subsection{Model Performance and Predictability}

Machine learning models demonstrated remarkable ability to predict certain health outcomes from integrated climate and socioeconomic data, whilst showing limited predictive power for others (Figure 1). Glucose metabolism emerged as the most predictable outcome, with the optimised XGBoost model achieving an $R^2$ of 0.611 (95\% CI: 0.582-0.640) on the test set. This high predictability suggests that glucose regulation is particularly sensitive to environmental and social determinants, potentially serving as an integrated biomarker of heat stress and vulnerability.

Cardiovascular parameters showed moderate predictability, with models achieving $R^2$ values of 0.141 (95\% CI: 0.118-0.164) for diastolic blood pressure and 0.115 (95\% CI: 0.089-0.141) for systolic blood pressure. Electrolyte balance, specifically potassium levels, demonstrated limited but significant predictability ($R^2$ = 0.071, 95\% CI: 0.048-0.094), whilst total cholesterol showed minimal association with climate-socioeconomic factors ($R^2$ = 0.023, 95\% CI: -0.003-0.049).

Random Forest models consistently outperformed other algorithms, likely due to their ability to capture complex interactions without assuming specific functional forms. The superior performance for glucose prediction persisted across multiple validation strategies, including temporal cross-validation and geographic hold-out sets, suggesting robust and generalisable patterns.

\subsection{Temporal Dynamics of Heat-Health Relationships}

Analysis of different exposure windows revealed distinct temporal patterns for heat-health associations (Figure 2). For glucose metabolism, model performance increased steadily with longer exposure windows, peaking at 21 days ($R^2$ = 0.611) before declining slightly for longer periods. This pattern suggests that cumulative heat exposure over approximately three weeks most strongly influences metabolic function, consistent with the time scale of glycaemic control reflected in markers like fructosamine.

Interestingly, cardiovascular outcomes showed more immediate responses, with optimal prediction windows of 7-14 days for blood pressure. This shorter time scale aligns with known physiological adaptation processes, including plasma volume expansion and autonomic adjustment that occur within days to weeks of heat exposure. The temporal analysis thus reveals distinct physiological mechanisms: rapid cardiovascular adaptation versus slower metabolic adjustment to thermal stress.

Seasonal stratification showed that heat-health relationships were strongest during transition periods (September-November and February-April) rather than peak summer, suggesting that population adaptation during sustained hot periods may reduce health impacts. This finding has important implications for early warning systems, indicating that the first heat events of the season may pose particular risks.

\subsection{Feature Importance and Mechanistic Insights}

SHAP analysis revealed a hierarchical structure of factors influencing heat-health relationships (Figure 3). For glucose prediction, the top five features were: 21-day maximum temperature (SHAP importance: 0.234), heat vulnerability index (0.156), temperature-age interaction (0.089), 21-day mean humidity (0.078), and income composite score (0.067). Notably, whilst climate variables dominated the top positions, socioeconomic factors comprised 44\% of total model importance, highlighting their crucial modifying role.

The analysis revealed several mechanistic insights. Temperature effects on glucose were non-linear, with a threshold around 28\degrees C above which impacts accelerated. Age amplified temperature effects, with individuals over 50 showing 2.3-fold greater glucose responses per degree of warming compared to younger adults. The temperature-BMI interaction was particularly pronounced in overweight individuals (BMI 25-30), who showed intermediate vulnerability—higher than normal weight but lower than obese individuals, possibly reflecting competing effects of thermal mass and metabolic efficiency.

Socioeconomic factors operated through multiple pathways. Housing quality directly modified exposure (inferior insulation, limited cooling) whilst also serving as a proxy for broader vulnerabilities (nutrition, healthcare access, occupational exposure). Income effects were mediated primarily through adaptive capacity—ability to modify behaviour, access cooling, and maintain hydration during heat events.

\subsection{Socioeconomic Amplification of Vulnerability}

The heat vulnerability index, integrating multiple socioeconomic dimensions, ranged from -650.5 to +0.5 across the population—a 650-fold variation that dramatically shaped heat-health impacts (Figure 4). Participants in the highest vulnerability quartile showed 3.9-fold greater glucose responses to heat exposure compared to the lowest quartile (8.2 mg/dL vs 2.1 mg/dL per \degrees C increase, $p$ < 0.001).

Decomposition of the vulnerability index revealed that housing quality contributed 42\% to overall vulnerability, followed by income level (31\%), healthcare access (27\%), education (18\%), and employment status (12\%). These factors showed significant interactions: the protective effect of higher income was diminished in areas with poor healthcare access, whilst education showed stronger protective effects in communities with better infrastructure.

Spatial mapping revealed stark geographic disparities in heat vulnerability across Johannesburg. Informal settlements in the south and northwest showed vulnerability indices 4.7 times higher than affluent northern suburbs. These high-vulnerability areas also experienced more severe heat exposure due to urban heat island effects, creating a "double jeopardy" of increased exposure and reduced adaptive capacity.

\subsection{Gender-Specific Responses}

Gender stratification revealed significant differences in heat-health relationships (Figure 5). Women showed substantially higher glucose sensitivity to heat (3.4 mg/dL per \degrees C) compared to men (2.1 mg/dL per \degrees C, $p$ < 0.001). This difference persisted after adjustment for age, BMI, and socioeconomic factors, suggesting intrinsic physiological differences in thermoregulation or metabolic responses.

Conversely, cardiovascular responses showed smaller gender differences, with men displaying slightly higher systolic blood pressure responses (1.8 mmHg vs 1.3 mmHg per \degrees C, $p$ = 0.043) but similar diastolic responses. The gender-temperature interaction was most pronounced in the 36-50 age group, where women's glucose responses were 67\% higher than men's, potentially reflecting hormonal influences on metabolism and thermoregulation.

Socioeconomic factors partially mediated gender differences. Women in informal settlements showed 2.1-fold higher heat vulnerability than men in the same areas, partly attributable to greater household responsibilities limiting adaptive behaviours and higher rates of household air pollution exposure from cooking.

\section{Discussion}

This study provides the first comprehensive analysis of heat-health relationships in a large African urban population using explainable machine learning techniques. Our findings reveal three critical insights: heat impacts on health are highly predictable from integrated environmental and social data; socioeconomic factors fundamentally shape physiological vulnerability to heat; and glucose metabolism serves as a sensitive, integrated biomarker of heat stress with potential for early warning applications.

\subsection{Implications for Understanding Heat-Health Mechanisms}

The exceptional predictability of glucose metabolism from climate and socioeconomic data ($R^2$ = 0.611) suggests that heat stress impacts metabolic function through multiple convergent pathways. Direct physiological mechanisms include increased metabolic demands of thermoregulation, altered insulin sensitivity due to heat-induced inflammatory responses, and disrupted circadian regulation of glucose homeostasis \citep{Kenny2018, Lim2018}. Our finding of a 21-day optimal exposure window aligns with the kinetics of glycaemic control markers and suggests that cumulative metabolic stress, rather than acute responses, drives health impacts.

The identification of housing quality as the dominant socioeconomic predictor (42\% contribution to vulnerability) extends beyond simple exposure modification. Poor housing conditions create multiple stressors—thermal discomfort, sleep disruption, indoor air pollution—that collectively challenge metabolic homeostasis \citep{Quinn2014, Tham2020}. The synergistic interaction between temperature and housing quality observed in our models suggests that interventions improving housing thermal performance could yield disproportionate health benefits.

Gender differences in heat-glucose relationships warrant particular attention. Women's higher metabolic sensitivity to heat (3.4 vs 2.1 mg/dL per \degrees C) may reflect several factors: differences in thermoregulatory physiology, including lower sweat rates and higher core temperature thresholds for vasodilation \citep{Gagnon2013}; hormonal influences on glucose metabolism and thermal perception \citep{Charkoudian2017}; and gendered social roles that may limit adaptive behaviours. These findings suggest that heat adaptation strategies must consider gender-specific vulnerabilities and constraints.

\subsection{Relevance to Climate Change Adaptation}

Our results have immediate implications for climate change adaptation in African cities. The high predictability of glucose responses to heat exposure suggests potential for developing early warning systems based on metabolic monitoring. Given the widespread availability of glucose testing and the growing use of continuous glucose monitors, heat-health surveillance systems could feasibly track population vulnerability in real-time. The 21-day lag we identified provides a valuable window for preventive interventions when heat events are forecast.

The 650-fold range in socioeconomic heat vulnerability across Johannesburg's population challenges uniform public health approaches to heat protection. Our vulnerability mapping identifies specific wards where targeted interventions—cooling centres, housing upgrades, healthcare outreach—would yield greatest benefits. The dominance of modifiable factors (housing, income, healthcare access) in determining vulnerability suggests substantial scope for adaptation through social policy alongside traditional health system responses.

The seasonal variation in heat-health relationships, with stronger associations during transition periods, indicates that adaptation messaging should emphasise early-season heat events. Public health campaigns typically focus on mid-summer extremes, but our findings suggest that September-November heat events, when physiological and behavioural adaptation is incomplete, may pose equal or greater risks.

\subsection{Methodological Contributions and Limitations}

This study demonstrates the value of explainable machine learning for environmental health research in data-rich but analytically complex settings. The SHAP framework enabled us to move beyond "black box" predictions to understand mechanisms and interactions, essential for translating findings into policy. The consistency of results across different algorithms and validation strategies strengthens confidence in the identified patterns.

However, several limitations merit consideration. Despite our large sample size, the study was confined to Johannesburg, and generalisability to other African cities with different climates, infrastructure, and social conditions requires investigation. The retrospective design limits causal inference, although the strength and consistency of associations, biological plausibility, and temporal relationships support causal interpretations. Whilst we integrated multiple data sources, exposure assessment at individual address level remains imperfect, particularly for informal settlements where microscale temperature variation is substantial.

The use of single time-point biomarker measurements for most participants precluded analysis of within-person responses to temperature variation. Longitudinal studies with repeated measurements would enable stronger causal inference and understanding of adaptation processes. Additionally, whilst we examined numerous biomarkers, we lacked data on some potentially important outcomes, including mental health, occupational injuries, and pregnancy outcomes, which may show distinct heat sensitivities.

\subsection{Future Research Directions}

Our findings open several avenues for future research. Mechanistic studies should investigate the biological pathways linking cumulative heat exposure to metabolic dysfunction, potentially including heat-induced changes in gut microbiota, adipose tissue function, and inflammatory signalling. The strong gender differences we observed warrant dedicated investigation of sex hormones, pregnancy, and menopause as modifiers of heat-health relationships.

Intervention studies could test whether glucose monitoring during heat events enables early detection and prevention of heat-related illness. Community-based trials of housing interventions—cool roofs, insulation, passive cooling—should assess impacts on both temperature exposure and health outcomes. The development of heat-health early warning systems tailored to African contexts, incorporating local vulnerability patterns and health system capacities, represents a critical translation opportunity.

Longer-term research should examine adaptation trajectories: how do heat-health relationships change with repeated exposure over months to years? Do populations show physiological adaptation, behavioural adjustment, or both? Understanding adaptation dynamics is crucial for projecting future health impacts under climate change scenarios.

\section{Conclusions}

This study provides robust evidence that heat-health impacts in African urban populations are predictable, mechanistic, and profoundly shaped by socioeconomic context. The exceptional sensitivity of glucose metabolism to cumulative heat exposure, amplified by social vulnerability, identifies both a biomarker for surveillance and a target for intervention. As African cities face escalating climate risks, our findings underscore the need for equity-centred adaptation strategies that address the root causes of vulnerability alongside immediate health threats.

The 650-fold variation in heat vulnerability across Johannesburg's population reveals the inadequacy of uniform public health responses to climate change. Effective adaptation requires targeted interventions that recognise how poverty, inadequate housing, and limited healthcare access transform environmental exposure into health inequity. The tools and insights developed in this study provide a foundation for evidence-based, equity-oriented climate adaptation in African cities and similar settings globally.

\section*{Acknowledgements}

We thank the participants who contributed their time and data to this research. We acknowledge the field teams, laboratory staff, and data managers across all participating cohorts. We are grateful to the South African Weather Service and SAAQIS for climate data access. This research was supported by grants from the Wellcome Trust (Grant 214207/Z/18/Z), the South African Medical Research Council, and the DSI-NRF Centre of Excellence in Human Development.

\section*{Author Contributions}

CP conceived the study, performed machine learning analyses, and drafted the manuscript. MC supervised the clinical aspects and contributed to interpretation. NC developed the analytical framework and reviewed statistical methods. SM, NF, and SM contributed to data collection and processing. AY, TK, and CW provided climate data and environmental health expertise. HH, NN, and DN contributed cohort data and reviewed the manuscript. All authors approved the final version.

\section*{Data Availability}

De-identified datasets and analysis code are available at \url{https://github.com/wrhi-heat-health} upon publication. Climate data are publicly available from ERA5 and SAAQIS. Individual-level health data are available upon reasonable request subject to ethical approval.

\section*{Competing Interests}

The authors declare no competing financial or non-financial interests.

\bibliographystyle{abbrvnat}
\bibliography{references}

\end{document}